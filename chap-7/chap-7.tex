\chapter{Preparing molecular dynamics calculations in \atomes}
\markboth{Preparing molecular dynamics calculations in \atomes}{\scshape \thesection}
\label{md}

\atomes\ proposes assistants dedicated to the creation/preparation/modification of input files for (massively parallel) molecular dynamics calculations: 
\begin{itemize}
\item Classical MD:
\begin{itemize}
\item \dlpoly\ \cite{DLPOLY}
\item \lammps\ \cite{LAMMPS}
\end{itemize}
\item {\em{ab-initio}} MD:
\begin{itemize}
\item \cpmd\ \cite{CPMD}
\item \cptk\ \cite{CP2K}
\end{itemize}
\item QM-MM (Mixed {\em{ab-initio}} - classical) MD:
\begin{itemize}
\item \cpmd\ \cite{CPMD}
\item \cptk\ \cite{CP2K}
\end{itemize}
\end{itemize}

\section{Classical MD}

\subsection{DL-POLY v4.09}

The \atomes\ helper for the \dlpoly\ calculation offers to prepare input files for {\bf{DL-POLY~v4.09}}. 
The {\bf{CONTROL}}, {\bf{FIELD}} and {\bf{CONFIG}} files can be entirely designed. \\
Before going further it is strongly advised to read the DL-POLY user manual: \\[0.25cm]

\href{ftp://ftp.dl.ac.uk/ccp5/DL\_POLY/DL\_POLY\_4.0/DOCUMENTS/USRMAN4.pdf}{ftp://ftp.dl.ac.uk/ccp5/DL\_POLY/DL\_POLY\_4.0/DOCUMENTS/USRMAN4.pdf}

\subsubsection{Initializing the force field}

\myfigure{t}{finit0}{\image{15}{img/md/dlpoly/init/init-0}}{Initializing the force field in the DL-POLY calculation assistant.}{Initializing the force field in the DL-POLY calculation assistant.}
The mandatory first step in the \dlpoly\ calculation assistant is to initialize the force field, as illustrated in figure~\ref{finit0} it required to select both: 
\begin{itemize}
\item The force field to be used, ie. the data source for many of the \aob{FIELD} file parameters including: bonds, angles, dihedrals and inversions. \\
The {\bf{FIELD}} file parameters can be imported from the following force fields: \\[0.35cm]
{\footnotesize{
\begin{minipage}{8cm}
\begin{itemize}
\item {\bf{OPLS force fields}}:
\begin{itemize}
\item OPLS All-atoms for proteins
\item OPLS All-atoms for RNA \\
\end{itemize}
\item {\bf{CHARMM force fields}}:
\begin{itemize}
\item CHARMM 22 proteins
\item CHARMM 22 proteins and metals
\item CHARMM 35 ethers
\item CHARMM 36 carbenes
\item CHARMM 36 general
\item CHARMM 36 lipids
\item CHARMM 36 Nucleid acids
\item CHARMM 36 proteins
\item CHARMM 36 proteins and metals
\item CHARMM silicates
\end{itemize}
\end{itemize}
\end{minipage}
\hspace{0.5cm}
\begin{minipage}{7cm}
\begin{itemize}
\item {\bf{AMBER force fields}}:
\begin{itemize}
\item AMBER 94
\item AMBER 96
\item AMBER 98
\item AMBER 99 \\
\end{itemize}
\item {\bf{Consistent force fields}}:
\begin{itemize}
\item CVFF
\item CVFF Augmented
\item CFF 91
\item PCFF
\item COMPASS
\end{itemize}
\end{itemize}
\end{minipage}
}}
\\[0.5cm]
Theses parameters are read by \atomes\ on the fly / by request from the corresponding files located in the directory: \mbox{\aob{bin\slash library\slash force\_fields}}. 
\item The atoms description to initiate the force field, ie. the first "rough" type of field atoms for the initial description of the force field:
\begin{itemize}
\item The atomic species
\item The atomic species and the total coordination(s)
\item The atomic species and the partial coordination(s)
\end{itemize}
\end{itemize}
Once the force field has been initialized the default pages populates the assistant: \\ 
\myfigure{h}{finit}{\image{13}{img/md/dlpoly/init/init-1}}{The initialized force field in the DL-POLY calculation assistant.}{The initialized force field in the DL-POLY calculation assistant.}

\subsubsection{The \aob{CONTROL} file}

When the force field has been initialized and providing the \aob{CONTROL} file is to be prepared, it is possible to adjust the corresponding options. \\
These options are regrouped in 8 tabs illustrated in figure~\ref{fcontrol}, 
and almost all options from the \dlpoly\ user guide are available, 
with the exception of the Plumed calculation options and some options for the two-temperature \aob{ttm} model.

\myfigure{h}{fcontrol}{\hspace{-0.5cm}\image{17}{img/md/dlpoly/control/control}}
{The \aob{CONTROL} file options in the DL-POLY calculation assistant.}
{The \aob{CONTROL} file options in the DL-POLY calculation assistant.}

\clearpage

\subsubsection{The \aob{FIELD} file}

If the \aob{FIELD} file is to be prepared then the corresponding tabs are also made available in the assistant. 

\subsubsection*{Force field elements}

The first tab related to the \aob{FIELD} file allows to enable or disable all kind of component(s) for the force field to be created: \\
\myfigure{h}{felem}{\image{16}{img/md/dlpoly/field/f1}}
{\aob{Select the component(s) of the force field} tab in the DL-POLY calculation assistant.}
{\aob{Select the component(s) of the force field} tab in the DL-POLY calculation assistant.}
\laf The \aob{Energy unit} combo box, allows to change the energy units for the field parameters, 
please note that changing the energy unit will convert all parameters in the \aob{FIELD} file to the new unit. \\[0.25cm]
The other options are divided in 2 categories, the \aob{Intra-molecular interaction(s)} options and the \aob{Non-bonded interaction(s)} options. 
To enable / disable any feature simply check / uncheck the corresponding button in this tab, then the associated page will 
be inserted / removed from the list of pages available in the assistant.

\clearpage
\subsubsection*{General behavior of the \aob{FIELD} file part of the assistant}

Each page of the assistant is dedicated to a particular property, the first pages (up to 13 pages) being dedicated to intra-molecular interactions 
and the latest pages (up to 6 pages) being dedicated to non-bonded interactions. \\[0.25cm]
Each page contains a table presenting the list of the associated properties,
double clicking on a line with the mouse left button will allow to edit the corresponding line/property, 
while the mouse right button will open a contextual menu with different set of actions including the edition available otherwise. \\[0.25cm]
For almost each and every intra or inter-molecular property listed in the corresponding tab, corresponds a \aob{Use} check button. 
It is mandatory to check-in the \aob{Use} button for the property to be used when building the \aob{FIELD} (remember that few are mutually exclusive), 
if not then the information is simply stored for further usage. \\[0.25cm]
Any objects selection is associated with a visualization in the model, ie. if an atom/a bond/an angle ... is selected then 
this object will be highlighted with a particular color in the 3D window, also the associated line in the selection dialog will be colored as well and using the same
color. \\[0.25cm]
Each page will be briefly introduced thereafter. 

\subsubsection*{The \aob{Molecule(s)} tab}

The \aob{Molecule(s)} tab, see figure~\ref{molinit}, lists all the different molecules in the model: \\
\myfigure{h}{molinit}{\image{16}{img/md/dlpoly/field/mol/mol-init}}
{\aob{Molecule(s)} tab in the DL-POLY calculation assistant.}
{\aob{Molecule(s)} tab in the DL-POLY calculation assistant.}
\clearpage
\noindent Each line corresponds to a molecule, and for each molecule the information is as follow: 
\begin{enumerate}
\item ID in the force field
\item Name in the force field (can be changed by double clicking on the line)
\item Multiplicity (number of times the same molecule appears in the model)
\item Chemistry
\item Total number of atoms
\item Number of chemical species
\item\label{vizm} Check button to visualize the molecule in the model
\item\label{vizn} Check button to visualize the atom number within the molecule
\end{enumerate}
In the \aob{Ni-Phth} example used here, two distinct molecules are found, 8 identical and standard Ni-phthalocyanine molecules 
that constitute the bottom surface and 1 modified Ni-phthalocyanine molecule, with a H atom missing on top of it. \\
As illustrated in figure~\ref{ivizm} the effect of the first check button \ref{vizm} is to help visualizing the different molecule(s) 
in the model. 
For each molecule a color is defined and applied to both the atom(s) in the model and the corresponding line in the \aob{Molecule(s)} tab. \\
\myfigure{h}{ivizm}{\hspace{-1cm}\image{17.5}{img/md/dlpoly/field/mol/mol-col}}
{Visualizing the \aob{Molecule(s)} using the DL\_POLY calculation assistant.}
{Visualizing the \aob{Molecule(s)} using the DL\_POLY calculation assistant.}
\laf Whereas as illustrated in figure~\ref{ivizn} the effect of the second check button \ref{vizn} is to help visualizing the atom numbers
as assigned in the \aob{FIELD} file, numbers range from 1 to the total number of atom(s) in the molecule. 
\newpage
\myfigure{h}{ivizn}{\hspace{-1cm}\image{17.5}{img/md/dlpoly/field/mol/mol-id}}
{Visualizing the atom(s) number(s) within the \aob{Molecule(s)}}
{Visualizing the atom(s) number(s) within the \aob{Molecule(s)}, 
to clarify the representation a single fragment from the \aob{NiPhth} example is displayed.}
\noindent Some mouse options are available in the \aob{Molecule(s)} tab:
\begin{itemize} 
\item Double click anywhere on the line to edit the name of the molecule.
\item Use the right click to open the contextual menu to edit the name/add/remove molecule(s).
\end{itemize}
As illustrated in figures~\ref{moladd} and \ref{molrem} it is possible to add molecule(s) to the force field, 
or to merge existing molecules from the force field into a single description:
\begin{itemize}
\item Providing that the multiplicity of a molecule is higher than 1, and therefore than several identical molecular fragment exist, 
it is possible to split the existing molecule, preserving the old description for the non-selected fragment(s) and creating a copy for the selected fragment(s). 
This process is illustrated in figure~\ref{moladd}.
\item Providing that some molecules in the force field have the same chemistry and connectivity it is possible to merge the description 
this two separates molecules as one. 
This process is illustrated in figure~\ref{molrem}.
\end{itemize}
\myfigure{h}{moladd}{\hspace{-0.5cm}\image{16.5}{img/md/dlpoly/field/mol/mol-split}}
{Adding molecules to the force field using the DL\_POLY calculation assistant.}
{Adding molecules to the force field using the DL\_POLY calculation assistant.}
\myfigure{h}{molrem}{\hspace{-0.5cm}\image{16.5}{img/md/dlpoly/field/mol/mol-merge}}
{Merging molecules from the force field using the DL\_POLY calculation assistant.}
{Merging molecules from the force field using the DL\_POLY calculation assistant.}

\clearpage

\subsubsection*{The \aob{Atom(s)} tab}

When molecules have been properly defined, the next tab allows to configure the atom(s) description, this has to be done for each molecule: \\ 
\myfigure{h}{atab}{\hspace{-0.5cm}\image{16.5}{img/md/dlpoly/field/atoms/atoms-init}}
{The \aob{Atom(s)} tab in the DL-POLY assistant.}
{The \aob{Atom(s)} tab in the DL-POLY assistant.}
\laf As illustrated in figure~\ref{atab} the first element to notice in the \aob{Atom(s)} tab is the combo box for the choice of the molecule to work on, 
that same combo box will be inserted in any tab dedicated to intra-molecular interactions. \\
\\Each line corresponds to a particular atom for the molecule specified above, and for each atom the information is as follow: 
\begin{enumerate}
\item ID in the molecule
\item Name in the molecule
\item Element
\item Mass
\item Charge
\item Number of frozen atoms for this type
\item Total number of atoms for this type
\item Check button to visualize the atom in the model
\end{enumerate}
\clearpage 
\myfigure{h}{a-add}{\hspace{-0.5cm}\image{16.5}{img/md/dlpoly/field/atoms/add-atoms}}
{Inserting a new atom description in the \aob{Atom(s)} tab.}
{Inserting a new atom description in the \aob{Atom(s)} tab.}
\myfigure{h}{a-rem}{\hspace{-0.5cm}\image{16.5}{img/md/dlpoly/field/atoms/merge-atoms}}
{Merging two existing atom descriptions in the \aob{Atom(s)} tab.}
{Merging two existing atom descriptions in the \aob{Atom(s)} tab.}
\clearpage
\noindent Before editing an atom property it might be necessary to add or remove some atom types from the existing description, 
these actions are available using the mouse right button contextual menu. 
As illustrated in figure~\ref{a-add} it is possible to create a new type of atom from an existing atom type that contains more that 1 atom. 
And as illustrated in figure~\ref{a-rem} it is possible to merge an existing atom description with any other atom type of the same chemical species. 
Whether it is for inserting or merging an atom it is important to specify that all existing atomic descriptions, a part for the new one or the 
one to be removed, will be preserved and thus that no information will be lost in this process. \\
\\
Editing an atom properties, by double clicking on the atom line or using the contextual menu, will open the \aob{Atom parameter(s)} dialog: \\
\myfigure{h}{a-edit}{\image{13.5}{img/md/dlpoly/field/atoms/edit-atoms}}{Editing an atom description in the \aob{Atom(s)} tab.}{Editing an atom description in the \aob{Atom(s)} tab.}
\laf The \aob{Field parameters} combo allows to switch between \aob{Manual} (user defined only) and \aob{Automatic} force field parameters (if any available). 
By default the choice is set on \aob{Manual} and the \aob{Force field type} combo is inactive. 
The \aob{Frozen} button allows to pick atom(s) to be frozen among the edited atom type (see figure~\ref{a-edit}). 
\clearpage
\noindent When selecting \aob{Automatic} for the \aob{Field parameters} combo, providing that some force field parameters are found \aob{Force field type} combo becomes active 
and offers to choose between available description(s): \\
\myfigure{h}{f-edit}{\image{13.5}{img/md/dlpoly/field/atoms/field-atoms}}
{Choosing a force field description for the atom type in the \aob{Atom(s)} tab.}
{Choosing a force field description for the atom type in the \aob{Atom(s)} tab. In this example available parameters are from the Amber 99 force field.}
\laf As soon as a selection has been made a dialog (see fig.~\ref{amb-at}) pops up to update the \aob{FIELD} file with the available parameters in the force field selected
at the initializing stage of the assistant (Amber 99 in this example). \\
%\clearpage
\myfigure{h}{amb-at}{\image{9.5}{img/md/dlpoly/field/atoms/amb-atoms}}
{The \aob{Update FIELD file with the force field parameters} dialog.}{The \aob{Update FIELD file with the force field parameters} dialog.}

\clearpage
\noindent The number of lines in the table in figure~\ref{amb-at} depends on both the option(s) (\aob{Flexible chemical bond(s), Bond angle(s) ...}) selected to create the \aob{FIELD} file, 
the line will appears only if the option was selected, and the availability of the data for that particular type of atom in the selected force field. 
In this example all the properties presented were selected to create the \aob{FIELD} file and some data exist in the force field for the \aob{CB} atom type with each particular property. 
\noindent This dialog helps to update quickly any parameters to be used to create the \aob{FIELD} file: 
\myfigure{h}{amb-li}{\image{16}{img/md/dlpoly/field/atoms/amb-ligne}}
{Updating parameters for the \aob{FIELD} file using the \aob{CB} atom(s) in the \aob{Amber 99} force field.}
{Updating parameters for the \aob{FIELD} file, using field parameters for the \aob{CB} atom(s) in the \aob{Amber 99} force field.}
\laf As illustrated in figure~\ref{amb-li} to update any \aob{FIELD} file value simply browse the property tree, select the desired force field parameters among available data, 
remember to check the \aob{update} button at the end of the line, and finally pressed the \aob{Apply} button.

\clearpage

\subsubsection*{\aob{Core-shell unit(s)}}
\vspace{-0.125cm}
\myfigure{h}{css}{\image{16}{img/md/dlpoly/field/intra/cs}}{\aob{Core-shell unit(s)} tab in the \aob{DL-POLY} assistant.}{\aob{Core-shell unit(s)} tab in the \aob{DL-POLY} assistant.}

\clearpage

\subsubsection*{\aob{Constraint bond(s)}}
\vspace{-0.125cm}
\myfigure{h}{cbs}{\image{16}{img/md/dlpoly/field/intra/cb}}{\aob{Constrains bond(s)} tab in the \aob{DL-POLY} assistant.}{\aob{Constrains bond(s)} tab in the \aob{DL-POLY} assistant.}

\clearpage

\subsubsection*{\aob{Mean force potential(s)}}
\vspace{-0.125cm}
\myfigure{h}{mpfs}{\image{16}{img/md/dlpoly/field/intra/mpf}}{\aob{Mean force potential(s)} tab in the \aob{DL-POLY} assistant.}{\aob{Mean force potential(s)} tab in the \aob{DL-POLY} assistant.}

\clearpage

\subsubsection*{\aob{Rigid unit(s)}}
\vspace{-0.125cm}
\myfigure{h}{rus}{\image{16}{img/md/dlpoly/field/intra/ru}}{\aob{Rigid unit(s)} tab in the \aob{DL-POLY} assistant.}{\aob{Rigid unit(s)} tab in the \aob{DL-POLY} assistant.}

\clearpage

\subsubsection*{\aob{Tethering potential(s)}}
\vspace{-0.125cm}
\myfigure{h}{teth}{\image{16}{img/md/dlpoly/field/intra/tet}}{\aob{Tethering potential(s)} tab in the \aob{DL-POLY} assistant.}{\aob{Tethering potential(s)} tab in the \aob{DL-POLY} assistant.}

\newpage
\subsubsection*{\aob{Flexible chemical bond(s)} and \aob{Bond restraint(s)}}

The tabs for the \aob{Flexible chemical bond(s)} and the \aob{Bond restraint(s)} being extremely similar, 
this manual presents only the part related to \aob{Flexible chemical bond(s)}. \\
As illustrated in figure~\ref{bds} the bonds, for the selected molecule, are listed by type depending on the nature of the field atoms involved, 
also for each type of bond it is possible to display every single bond in the molecule. \\
\myfigure{h}{bds}{\image{16}{img/md/dlpoly/field/bonds/bonds}}{\aob{Flexible chemical bond(s)} tab in the \aob{DL-POLY} assistant.}{\aob{Flexible chemical bond(s)} tab in the \aob{DL-POLY} assistant.}
\laf Each line corresponds to a either to a bond type, or a particular bond, for the molecule specified above, and for each bond the information is as follow:
\begin{enumerate}
\item ID in the molecule (if the line refers to a bond type).
\item Name for atom type 1 (bond type), or field number for atom 1 (bond).
\item Name for atom type 2 (bond type), or field number for atom 2 (bond).
\item Value measured in the model: average for all bonds for that type, or, value (or average if the multiplicity for the molecule is > 1) for that particular bond.
\item Visualize all bonds of that type or that particular bond in the model.
\item Use to create the \aob{FIELD} file, if not checked the bond / type will not be used.
\item Potential to use to describe all bonds of that type (\aob{Default:} value) or that particular bond in the model.
\item Numerical value(s) for the parameters of the selected potential.
\end{enumerate}
The description of the bond(s) can be modified simultaneously for all bond(s) of the same type using the root line that type, 
or one by one by browsing the list and picking any line. 
It is therefore possible to choose a different parametrization for every single bond in the molecule. 
Some parameters can be modified directly in the table, 
otherwise a simple double click using the mouse left button will open the bond edition dialog illustrated in figure~\ref{bded}. 
\myfigure{h}{bded}{\image{15}{img/md/dlpoly/field/bonds/bd-edit}}
{\aob{Flexible chemical bond(s)} edition dialog in the \aob{DL-POLY} assistant.}{\aob{Flexible chemical bond(s)} edition dialog in the \aob{DL-POLY} assistant.}
\clearpage
\noindent In figure~\ref{bded} the \aob{Field parameters} combo box presents the list of matching parameters, chemical species wise, 
found in the force field selected at the initialization stage. \\[0.5cm]
The next tabs of the assistant dedicated to intra-molecular properties are as follow:
\begin{itemize}
\item \aob{Bond angle(s)}
\item \aob{Angular restraint(s)}
\item \aob{Dihedral angle(s)}
\item \aob{Torsional restraint(s)}
\item \aob{Improper angle(s)}
\item \aob{Inversion angle(s)}
\end{itemize}
For all these properties the philosophy is similar to the one presented above for the \aob{Flexible chemical bond(s)}. 
Hence will only briefly illustrate the different tabs without more details. 

\subsubsection*{\aob{Bond angle(s)} and \aob{Angular restraint(s)}}

\myfigure{h}{ans}{\image{16}{img/md/dlpoly/field/intra/angles}}{\aob{Bond angle(s)} tab in the \aob{DL-POLY} assistant.}{\aob{Bond angle(s)} tab in the \aob{DL-POLY} assistant.}
\newpage

\subsubsection*{\aob{Dihedral angle(s)} and \aob{Torsional restraint(s)}}

\myfigure{h}{dihs}{\image{16}{img/md/dlpoly/field/intra/dih}}{\aob{Dihedral angle(s)} tab in the \aob{DL-POLY} assistant.}{\aob{Dihedral angle(s)} tab in the \aob{DL-POLY} assistant.}
\vspace{-0.5cm}

\subsubsection*{\aob{Improper angle(s)}}

\myfigure{h}{imps}{\image{16}{img/md/dlpoly/field/intra/imp}}{\aob{Improper angle(s)} tab in the \aob{DL-POLY} assistant.}{\aob{Improper angle(s)} tab in the \aob{DL-POLY} assistant.}
\newpage

\subsubsection*{\aob{Inversion angle(s)}}

\myfigure{h}{invs}{\image{16}{img/md/dlpoly/field/intra/inv}}{\aob{Inversion angle(s)} tab in the \aob{DL-POLY} assistant.}{\aob{Inversion angle(s)} tab in the \aob{DL-POLY} assistant.}

\clearpage

\subsubsection*{\aob{van der Waals potential(s)}}
\vspace{-0.125cm}
\myfigure{h}{vdws}{\image{16}{img/md/dlpoly/field/inter/vdw}}{\aob{van der Waals potential(s)} tab in the \aob{DL-POLY} assistant.}{\aob{van der Waals potential(s)} tab in the \aob{DL-POLY} assistant.}

\clearpage

\subsubsection*{\aob{Metal potential(s)}}
\vspace{-0.125cm}
\myfigure{h}{mets}{\image{16}{img/md/dlpoly/field/inter/met}}{\aob{Metal potential(s)} tab in the \aob{DL-POLY} assistant.}{\aob{Metal potential(s)} tab in the \aob{DL-POLY} assistant.}

\clearpage

\subsubsection*{\aob{Tersoff potential(s)}}
\vspace{-0.125cm}
\myfigure{h}{ters}{\image{16}{img/md/dlpoly/field/inter/ter}}{\aob{Tersoff potential(s)} tab in the \aob{DL-POLY} assistant.}{\aob{Tersoff potential(s)} tab in the \aob{DL-POLY} assistant.}

\clearpage

\subsubsection*{\aob{Three body potential(s)}}
\vspace{-0.125cm}
\myfigure{h}{tbds}{\image{16}{img/md/dlpoly/field/inter/tbd}}{\aob{Three potential(s)} tab in the \aob{DL-POLY} assistant.}{\aob{Three potential(s)} tab in the \aob{DL-POLY} assistant.}

\clearpage

\subsubsection*{\aob{Four body potential(s)}}
\vspace{-0.125cm}
\myfigure{h}{fbds}{\image{16}{img/md/dlpoly/field/inter/fbd}}{\aob{Four body potential(s)} tab in the \aob{DL-POLY} assistant.}{\aob{Four body potential(s)} tab in the \aob{DL-POLY} assistant.}

\clearpage

\subsubsection*{\aob{External field(s)}}
\vspace{-0.125cm}
\myfigure{h}{fexts}{\image{16}{img/md/dlpoly/field/inter/fext}}{\aob{External field(s)} tab in the \aob{DL-POLY} assistant.}{\aob{External field(s)} tab in the \aob{DL-POLY} assistant.}


\clearpage

\subsubsection{\aob{Preview}}

At any time when using the \aob{DL-POLY} assistant, the \aob{Preview} button allows open the \aob{DL-POLY files preview} dialog, to visualize the \aob{CONTROL}, \aob{FIELD} and \aob{CONFIG} file(s). \\
\myfigure{h}{prevdlp}{\image{16}{img/md/dlpoly/preview}}
{\aob{DL-POLY files preview} dialog in the \aob{DL-POLY} assistant.}{\aob{DL-POLY files preview} dialog in the \aob{DL-POLY} assistant.}
\laf The \aob{DL-POLY files preview} dialog (fig.~\ref{prevdlp}) contains a notebook with up to 3 tabs depending on the file(s) to be created. 
Each tab presents the content of a file, \aob{CONTROL}, \aob{FIELD} or \aob{CONFIG}, and highlights option(s) and keyword(s) using color(s) and font layout option(s). 

\subsubsection{Finalizing the assistant and file(s) creation}

When pressing the \aob{Apply} button on the file tab of the assistant, the user is asked to selected a folder. 
\aob{CONTROL}, \aob{FIELD} and \aob{CONFIG} if selected, will be created in the selected directory. \\[0.25cm]
Following the file(s) creation or if the assistant is closed anyway, the \aob{DL-POLY} force field data will be 
preserved. 
Therefore allowing the re-open the assistant later for further modifications. \\[0.25cm]
All existing force field data can also saved in the \atomes\ project and/or workspace files, simply remember to save your work. 


\newpage
\subsection{LAMMPS}

{\em{Coming soon}}

\newpage

\section{{\em{Ab-initio MD}}}

\subsection{CPMD v4.3.0}

The \atomes\ helper for the \cpmd\ calculation offers to prepare input files for {\bf{CPMD~v4.3.0}}. 
Please note that the \cpmd\ code offers so many calculation options that it is not possible either to provide a description 
or to offer a comprehensive usage guide for each of these options.  
Therefore the \cpmd\ calculation assistant only provides help towards basics and / or frequently used calculation options. \\[0.25cm]
In any case if you intent to use the \cpmd\ code please refer to the user manual: \\[0.25cm]
\href{https://www.cpmd.org/wordpress/CPMD/getFile.php?file=manual.pdf}{https://www.cpmd.org/wordpress/CPMD/getFile.php?file=manual.pdf}\\[0.25cm]
The \atomes\ helper for the \cpmd\ calculation provides a step by step interface to configure the different sections of the \cpmd\ input file:

\subsection*{The \aob{INFO} section}

\myfigure{h}{cinfo}{\image{16}{img/md/cpmd/info}}{\aob{INFO section} tab in the \aob{CPMD} assistant.}{\aob{INFO section} tab in the \aob{CPMD} assistant.}

\newpage

\subsection*{The \aob{CPMD} section}

The creation of the \aob{CPMD} section is split over 3 tabs, dedicated respectively to the calculation, thermostat and restart option(s). 
\subsubsection*{\aob{Calculation options}}

The \aob{Calculation to be performed} [Fig.~\ref{ccpmd1}] combo allows to select the type of job: \\
\myfigure{h}{ccpmd1}{\image{16}{img/md/cpmd/cpmd}}{\aob{CPMD section - Calculation options} tab in the \aob{CPMD} assistant.}{\aob{CPMD section - Calculation options} tab in the \aob{CPMD} assistant.}
\laf The option(s) directly bellow the combo will change depending on the nature of the calculation, please refer
to the \cpmd\ user manual for more information on theses options. 

\newpage

\subsubsection*{\aob{Thermostat options}}

If a molecular dynamics calculation is to be performed then the \aob{CPMD section - Thermostat options} tab becomes accessible:  
\myfigure{h}{ccpmd2}{\image{16}{img/md/cpmd/thermo}}{\aob{CPMD section - Thermostat options} tab in the \aob{CPMD} assistant.}{\aob{CPMD section - Thermostat options} tab in the \aob{CPMD} assistant.}
\laf In order to access the next page of the assistant thermostat(s) must be set up properly. 
Again for more information please refer to the \cpmd\ user manual. 

\subsubsection*{\aob{RESTART options}}

The \aob{CPMD section - Restart options} tab present options to restart calculation and saving options during the calculation: 
\myfigure{h}{ccpmd3}{\image{16}{img/md/cpmd/restart}}{\aob{CPMD section - Restart options} tab in the \aob{CPMD} assistant.}{\aob{CPMD section - Restart options} tab in the \aob{CPMD} assistant.}

\newpage

\subsection*{The \aob{DFT} section}

\myfigure{h}{cdft}{\image{16}{img/md/cpmd/dft}}{\aob{DFT section} tab in the \aob{CPMD} assistant.}{\aob{DFT section} tab in the \aob{CPMD} assistant.}

\subsection*{The \aob{VDW} section}

\myfigure{h}{cvdw}{\image{16}{img/md/cpmd/vdw}}{\aob{VDW section} tab in the \aob{CPMD} assistant.}{\aob{VDW section} tab in the \aob{CPMD} assistant.}
\noindent Providing that van der Waals interactions are selected in the \aob{CPMD} section of the input file, 
then the \aob{VDW} section becomes accessible. 
However its content is filled automatically based on parameters selected in both the \aob{CPMD} and \aob{DFT} sections.
\newpage

\subsection*{The \aob{PROP} section}

\myfigure{h}{cprop}{\image{16}{img/md/cpmd/prop}}{\aob{PROP section} tab in the \aob{CPMD} assistant.}{\aob{PROP section} tab in the \aob{CPMD} assistant.}
\noindent Providing that the \aob{Calculation of physical properties} is selected in the \aob{CPMD} section of the input file, 
then the \aob{PROP} section becomes accessible. 
However its content is filled automatically based on parameters selected in the \aob{CPMD} section.

\subsection*{The \aob{SYSTEM} section}

\myfigure{h}{csyst}{\image{16}{img/md/cpmd/system}}{\aob{SYSTEM section} tab in the \aob{CPMD} assistant.}{\aob{SYSTEM section} tab in the \aob{CPMD} assistant.}

\newpage

\subsection*{The \aob{ATOMS} section}

\myfigure{h}{cats}{\image{16}{img/md/cpmd/atoms}}{\aob{ATOMS section} tab in the \aob{CPMD} assistant.}{\aob{ATOMS section} tab in the \aob{CPMD} assistant.}

\newpage

\subsection{CP2K}

The \atomes\ helper for the \cptk\ calculation offers to prepare input files for {\bf{CP2K~v9.1}}. 
Please note that the \cptk\ code offers so many calculation options that it is not possible either to provide a description 
or to offer a comprehensive usage guide for each of these options.  
Therefore the \cptk\ calculation assistant only provides help towards basics and / or frequently used calculation {\it{ab-initio}} options. \\[0.25cm]
In any case if you intent to use the \cptk\ code please refer to the user manual: \\[0.25cm]
\href{https://www.cp2k.org/howto}{https://www.cp2k.org/howto}\\[0.25cm]
The \atomes\ helper for the \cptk\ calculation provides a step by step interface to configure the different sections of the \cptk\ input file(s) for {\it{ab-initio}} calculations:

\subsection*{The \aob{CP2K input structure} section}

The CP2K input can be rather initimidating, you can decide at this stage either to have all input data gathered in a single file, 
or alternatively split in several, content orientated, files:  


\subsection*{The \aob{GLOBAL} section}

Select the calculation to be performed, and some general options regarding the process. \\
Note that \atomes\ provides, when available, basis set(s) and pseudo-potential(s) from the CP2K database. 
Basis set(s) and pseudopotential(s) will be saved respectivelly in \aob{basis.inc} and \aob{pseudo.inc}. 

\subsection*{The \aob{FORCE\_EVAL} section}

Describe the details of the forces evaluation process.

\subsection*{The \aob{SUBSYS} section}

\subsection*{The \aob{MOTION} section}

\section{Quantum Mechanics and Molecular Mechanics MD}

\subsection{CPMD}

{\em{Coming soon}}

\subsection{CP2K}

{\em{Coming soon}}

