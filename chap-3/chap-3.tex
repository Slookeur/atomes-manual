\chapter{Features}
\markboth{Features}{\scshape \thesection}

\section{Main window}

\mainfig
The main interface of the \atomes\ program [Fig.~\ref{main}-a] gives access to:
\begin{itemize}
\item The \aob{Workspace menu} [Fig.~\ref{main}-b] to open and save files, including \atomes\ workspace and project files.
\item The \aob{Edit menu} [Fig.~\ref{main}-c] to adjust chemical / physical of a system to be studied/edited.
\item The \aob{Compute menu} [Fig.~ \ref{main}-d] to run some analysis on the \activp\ project (see section \ref{calc} for details).
\item The \aob{Help menu} [Fig.~\ref{main}-e] to access the documentation and a periodic table.
\end{itemize}

\atomes\ allows to build, edit, study, analyse and compare multiple systems within the same instance of the program. 
Each system (molecule(s)/material ...) is defined in a data structure called {\bf{project}}, \atomes\ projects (*.apf files) can be opened/saved separately, 
the list of opened project(s) appears in the left side of the main window in a browsable tree-like structure. 
The entire list of project(s) displayed in this part of the main window is a data structure called {\bf{workspace}}, 
and \atomes\ workspace files (*.awf files) can also be opened and saved, allowing to back up the entire list of projects at once, 
preserving the possible connections, data exchanges, between each projects. \\
An illustration is presented in figure~\ref{overview}, many different projects being opened in the workspace. 
To each of these project is assigned an OpenGL 3D window, few of them being visible in figure~\ref{overview}.  
Double-clicking on the \aob{Workspace} word at the top root of the tree, the list of all projects appears in the right side window. 
This list includes, in bold green font, the name of the \activp\ project (see section~\ref{calc} for details).

\subsection*{Keyboard shortcuts}
\vspace{0.5cm}
\kbdmain
\clearpage

\section{Workspace and project tree}

In \atomes\ each project, as soon as it is created and whether it is new (empty) or contains atomic coordinates, is immediately:
\begin{itemize}
\item inserted into the workspace tree: a branch will appear bearing the project name [Fig.~\ref{ptree}].
\item assigned an OpenGL window for visualization, analysis and edition. 
\item the \aob{Toolboxes} dialog [Fig.~\ref{tools}] is refreshed, and if required open, to present the data of this project.
\end{itemize}
Figure \ref{ptree} illustrates the structure of project tree branch in the \atomes\ workspace. \\
\treefig
\laf On each line/tree branch in the workspace tree the double-click with the left button of the mouse has an effect: 
\begin{itemize}
\item \aob{Workspace} line: displays workspace information, lists all opened project(s) and gives the name of the \activp\ project in bold green font [Fig.~\ref{overview}].
\item \aob{Project's name} line: activates the project (see section~\ref{calc} for details), when the project is \activp\ its name is displayed in bold font.
\item \aob{Settings} line: provides general information about the project.
\item \aob{Visualization} line: open/close the OpenGL visualization window, and display OpenGL/hardware information. 
\item Each of the calculation lines (ex: \aob{g(r)/G(r)}): if the calculation was performed for that project displays summary of the calculation, used parameters, and results. 
\end{itemize}
As illustrated in figure~\ref{wlc} the right click of the mouse display a menu basically reproduces the Workspace menu [Fig.~\ref{main}-b], 
but also introduces two new buttons to activate the project (see section~\ref{calc} for details) and edit its name.\\ 
\myfigure{h}{wlc}{\image{8}{img/p-tree/atomes-wlc}}{Mouse left click menu in the workspace tree of the \atomes\ program.}{Mouse left click menu in the workspace tree of the \atomes\ program.}
\laf The name of the project supports Pango markup\cite{pangomarkup}: \\
\\
{\small{
\begin{tabular}{p{5cm}p{5cm}p{5cm}}
$\bullet$ <sub> </sub> for $_{\text{underscore}}$ & $\bullet$ <i> </i> for {\it{italic}} & $\bullet$ <u> </u> for {\uline{underline}} \\[0.25cm]
$\bullet$ <sup> </sup> for $^{\text{exponent}}$ & $\bullet$ <b> </b> for {\bf{bold}} & $\bullet$ <tt> </tt> for {\texttt{monospace}} \\[0.25cm]
\multicolumn{3}{l}{and more ... see: \href{https://docs.gtk.org/Pango/pango\_markup.html}{https://docs.gtk.org/Pango/pango\_markup.html}}\\[0.25cm]
\end{tabular}
}}

\section{Files}

\subsection{Importing atomic coordinates}
\label{import}

The list of the supported format of atomic coordinates is presented in appendix~\ref{inout}. \\ 
\href{https://isaacs.sourceforge.net}{ISAACS} files contains all the information required to prepare analysis and visualization. 
For the other file formats, and after reading the atomic coordinates, dialog boxes appears automatically in the following order:
\subsubsection*{1) The \aob{Chemistry and physics} dialog [Fig.~\ref{cpd}]}
\myfigure{h}{cpd}{\image{8}{img/main/atomes-cpd}}
{The \aob{Chemistry and physics} dialog in the \atomes\ program.}{The \aob{Chemistry and physics} dialog in the \atomes\ program.}
This dialog allows to tweak the chemical and physical properties of the elements found in the coordinates file (see figure~\ref{cpd}). \\
Particular attention should be given to the selection of the neutrons and X-rays scattering length, 
for the former parameters are included in \atomes\ (see chapter~\ref{datab}), 
for the latter two possibilities are offered to the user:
\begin{itemize}
\item Use the exact Q dependent method to compute X-rays S(Q) related properties.
\item Use an approximation with the X-rays scattering length equal to the atomic number of the element.
\end{itemize}
%\clearpage
\subsubsection*{2) The \aob{Box and periodicity} dialog [Fig.~\ref{bpd}]}
\bpdfig
This dialog allows to define the periodicity and if needed adjust the model box proportions, or lattice vectors (see [Fig.~\ref{bpd}]). 
\subsubsection*{3) The \aob{Bond cutoffs} dialog [Fig.~\ref{bcd}]}
\myfigure{h}{bcd}{\image{8}{img/main/atomes-bcd}}{The \aob{Bond cutoffs} dialog in the \atomes\ program.}{The \aob{Bond cutoffs} dialog in the \atomes\ program.}
This dialog allows to adjust the cutoff distances used to define the existence or absence of chemical bonds (see figure~\ref{bcd}), 
both for the calculations (see section~\ref{calc}) sensitive to these parameters and for the 3D visualization in the OpenGL window (see section~\ref{visu}). \\ 
%\clearpage
As soon as the \aob{Bond cutoffs} dialog closes \atomes\ will have everything required to setup the 3D model and the OpenGL window will appear. \\
\\
The 3 dialog boxes \aob{Chemistry and physics}, \aob{Box and periodicity} and \aob{Bond cutoffs} can be re-opened again later on using the \aob{Edit} menu [Fig.~\ref{main}-c]. 
However in that case only the \activp\ project parameters can to be modified. Therefore remember to activate the project you want to edit before using the edit menu.
\clearpage

\subsection{Reading \atomes\ project file(s)}
\label{apf}

The \atomes\ project file allows to store:
\begin{itemize}
\item The atomic coordinates, including MD trajectories.
\item The results/data of all calculations performed within \atomes. 
\item The results of all modifications of the calculations data, including graph windows (see chapter~\ref{ana}). 
\item The parameters of the OpenGL window, so that when re-opened the project appears exactly as it was when saved. 
\end{itemize}
The main idea being to be able to resume work exactly where it was before saving the \atomes\ project file. \\
\\The \atomes\ project files have the extension: {\bf{.apf}} \\
\\
To open \atomes\ project file(s) use the \aob{Open Project File(s)} dialog [Fig.~\ref{oapf}]. \\
This can be done using alternatively: 
\begin{itemize}
\item The workspace menu.
\item The right click menu obtained with the mouse button of the workspace tree.
\item The keyboard shortcut \Ctrl + \keystroke{o} on top of the \atomes\ program main window.
\end{itemize}
Many \atomes\ project files can be opened simultaneously, simply select all projects to be opened in the \aob{Open Project File(s)} dialog, click \aob{Open} and they will appear in the workspace tree: \\
\oapfig
\laf Later on when using the \atomes\ program remember that any window specifically dedicated to a project will have the name of this project in its title bar. 
\clearpage

\subsection{Reading \atomes\ workspace file(s)}
\label{awf}

The \atomes\ workspace file allows to store:
\begin{itemize}
\item A collection of \atomes\ projects. 
\item The relations between these projects, data exchanges, comparisons ...
\end{itemize}
Again the idea is to be able to resume work exactly where it was before saving the \atomes\ workspace file. \\
\\The \atomes\ workspace files have the extension: {\bf{.awf}}\\
\\
To open \atomes\ workspace file use the \aob{Open Workspace} dialog [Fig.~\ref{owpf}]. \\
This can be done using alternatively: 
\begin{itemize}
\item The workspace menu.
\item The right click menu obtained with the mouse button of the workspace tree.
\item The keyboard shortcut \Ctrl + \keystroke{w} on top of the \atomes\ program main window. 
\end{itemize}
Only a single \atomes\ workspace file can be opened at a time, if needed close the opened workspace, then open the new one: 
\owpfig

\clearpage

\section{Analyzing models using \atomes}
\label{calc}

\atomes\ can compute the following structural characteristics of a 3D structure model:
{\small{
\begin{itemize}
\item Radial distribution functions g(r) (RDFs) \cite{AlTilde} including $^{\circ}$:
\begin{itemize}
\item Total RDFs for neutrons and X-rays.
\item Partial RDFs.
\item Bhatia-Thornton RDFs \cite{2007JNCS..353.2959S}
\end{itemize}
$^{\circ}$ Radial distribution functions can be computed by i) direct real space calculation and/or ii) Fourier transforming of the structure factor calculated using the Debye formalism \cite{EurJM.14-233.2002}
\item Structure factors S(q) \cite{EurJM.14-233.2002} including $^{\circ\circ}$:
\begin{itemize}
\item Total structure factors S(q) for neutrons and X-rays.
\item Total Q(q) \cite{EurJM.14-233.2002,jac.17-61.1984} for neutrons and X-rays.
\item Partial S(q):
\begin{itemize}
\item Faber-Ziman \cite{PhilMag.11.153} partial S(q)
\item Ashcroft-Langreth \cite{PhysRev.156.685,PhysRev.159.500,PhysRev.166.934.2} partial S(q)
\item Bhatia-Thornton \cite{PhysRevB.2.3004} partial S(q)
\end{itemize}
\end{itemize}
$^{\circ\circ}$ Structure factors can be computed by i) Fourier transforming of the radial distribution functions and/or ii) using the Debye formalism \cite{EurJM.14-233.2002}
\item Interatomic bond properties
\begin{itemize}
\item Coordination numbers
\item Atomic near neighbor distribution
\item Fraction of links between tetrahedra
\item Fraction of tetrahedral units
\item Bond lengths distribution for the first coordination sphere
\end{itemize}
\item Distribution of Bond angles
\item Distribution of Dihedral angles
\item Ring statistics, according to several definitions:
\begin{itemize}
\item All closed paths (no rules)
\item King's rings \cite{Nature.213.1112,PhysRevB.44.4925}
\item Guttman's rings \cite{Guttman-116-145}
\item Primitive rings \cite{Goetzke-127.215,YuanCormack-24-343} (or Irreducible \cite{Wooten-bk0109})
\item Strong rings \cite{Goetzke-127.215,YuanCormack-24-343}
\end{itemize}
And including options to:
\begin{itemize}
\item search only for ABAB rings
\item exclude rings with homopolar bonds (A-A or B-B) from the analysis
\end{itemize}
Ring statistics is presented according to the R.I.N.G.S. method \cite{RINGS}.
\item Chain statistics, including options to:
\begin{itemize}
\item search only for AAAA chains
\item search only for ABAB chains
\item exclude chains with homopolar bonds (A-A or B-B) from the analysis
\item search only for 1-(2)$_n$-1 chains
\end{itemize}
\item Spherical harmonics invariant, $Q_l$, as local atomic ordering symmetry identifiers \cite{PhysRevB.28.784}
\begin{itemize}
\item Average Q$_l$ for each chemical species
\item Average Q$_l$ for a user specified structural unit
\end{itemize}
%\item Bond valence sums \cite{BondValence, ActaCryst.B41.244, ActaCryst.A29.266}
%\begin{itemize}
%\item Average bond valence for each chemical species
%\item Average bond valence for a user specified structural unityelke
%\end{itemize}
\item Mean Square Displacement of atoms (MSD)
\begin{itemize}
\item Atomic species MSD
\item Directional MSD (x, y, z, xy, xz, yz)
\item Drift of the center of mass
\end{itemize}
\end{itemize}
}}
See appendix \ref{physics} to learn more about the physics and the chemistry behind these calculations. \\
\\
The calculations presented in this list can only be performed on the \activp\ project, ie. the project which 
name appears in the title bar of the \atomes\ program main window, in bold font in the \atomes\ workspace tree and in green bold font in the workspace information dialog [Fig.~\ref{overview}]. \\
For more about running calculation using \atomes\ see chapter~\ref{ana}.

\clearpage

\section{Visual analysis using \atomes}
\label{visu}

Each model in the \atomes\ workspace is assigned an OpenGL window that provides an interactive experience to visualize and analyze its properties. 
Isolated configurations as well as entire molecular dynamics trajectories can be visualized. \\ 
Among the possibilities of the OpenGL window: 
\begin{itemize}
\item Advanced color map options, for both atoms and polyhedra. 
\item Advanced coordination(s) visualization options (total, partial, fragments, molecules). 
\item Advanced coordination polyhedra visualization options. 
\item Measurement tools.
\item Advanced layout system for atomic labels, measurements, MD box and model axis. 
\item Image and movie rendering, including an intuitive interface to movie encoding that allows to record every interaction/modification made to the OpenGL window. 
\item Advanced OpenGL configuration options.
\end{itemize}
For a complete description of these features see chapter~\ref{visual}. 

\section{Visual edition and model creation using \atomes}
\label{edit}

Using \atomes\ \aob{Crystal builder} it is possible to build crystalline structures or even super-structures.  
Also if the model description contains a box, then cell edition options become available:
\begin{itemize}
\item Wrap atoms in original cell
\item Shift cell center
\item Add extra cell(s)
\item Create super cell
\item Change the model density
\item Cut slabs or extract atoms from the model.
\end{itemize}
And if the model contains only a single configuration (and not a molecular dynamics trajectory), or when an empty project is created, the atom(s) edition options/mode become available:
\begin{itemize}
\item Motion (random or selected atom(s))
\item Replacement (random or selected atom(s))
\item Removal (random or selected atom(s))
\item Insertion
\end{itemize}
For a complete description of these features see chapter~\ref{edition}.

\section{Preparing MD calculations in \atomes}

\atomes\ provides input creation assistants for well known molecular dynamics codes:
\begin{itemize}
\item \cpmd\ \cite{CPMD}
\item \cptk\ \cite{CP2K}
\item \dlpoly\ \cite{DLPOLY}
\item \lammps\ \cite{LAMMPS}
\end{itemize}
For a complete description of these features see chapter~\ref{md}.

