\chapter{Introduction}

\atomes\ is a cross-platform Free (Open Source) computational material science tool box. 
It regroups a comprehensive panel of specialized edition tools, several assistants dedicated to the preparation of numerical experiments, 
and, a large number of physico-chemical analysis. 
Advanced visualization utilities and capabilities being available at each stage of the process \atomes\ 
introduces innovative 3D rendering possibilities and intuitive applications of the calculation results. \\[0.25cm]
\atomes\ is designed to analyze, to visualize and to create/edit large three-dimensional atomic scale models, 
and can handle MD trajectories from hundreds of thousands, up to millions, of atoms. 
Using OpenMP parallelization, file processing, coordination and physico-chemical analysis 
make use of the advantages of modern CPUs and their multiple cores. 
Consequently when importing atomic coordinates several properties are analyzed on the fly regardless of the size, 
in number of atoms or MD steps, and the periodicity of the system. \\
\atomes\ offers a workspace that allows to have many projects opened simultaneously. 
The different projects in the workspace can exchange data: analysis results, atomic coordinates ... \\[0.25cm]
\atomes\ also provides an advanced input preparation system for further calculations using well known molecular dynamics codes: 
\begin{itemize}
\item Classical MD : \dlpoly\ \cite{DLPOLY} and \lammps\ \cite{LAMMPS}
\item {\em{ab-initio}} MD : \cpmd\ \cite{CPMD} and \cptk\ \cite{CP2K}
\item QM-MM MD : \cpmd\ \cite{CPMD} and \cptk\ \cite{CP2K}
\end{itemize}
\noindent To prepare the input files for these calculations is likely to be the key, and most complicated step towards MD simulations. 
\atomes\ offers a user-friendly assistant to help and guide the user step by step to achieve this crucial step. 
\begin{figure}[!p]
\hypertarget{overview}{
\begin{center}
\includegraphics*[width=23cm, angle=90, keepaspectratio=true, draft=\ddst]{img/overview4}
\end{center}
\caption{Overview of the \atomes\ program.}\label{overview}
}
\end{figure}
