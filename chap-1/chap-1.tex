\chapter{Introduction}

\atomes\ is a cross-platform program developed around 3 main research ideas: to analyze, 
to visualize and to edit/create three-dimensional atomistic models. \\
\atomes\ offers a workspace that allows to have many projects opened simultaneously. 
The different projects in the workspace can exchange data: analysis results, atomic coordinates ... \\
By regrouping advanced analysis techniques, 3D visualization and 3D edition \atomes\ introduces innovative 3D rendering possibilities 
and intuitive utilizations of the calculation results. \\[0.25cm]
\atomes\ also provides an advanced input preparation system for further calculations using well known molecular dynamics codes: 
\begin{itemize}
\item Classical MD : \dlpoly\ \cite{DLPOLY} and \lammps\ \cite{LAMMPS}
\item {\em{ab-initio}} MD : \cpmd\ \cite{CPMD} and \cptk\ \cite{CP2K}
\item QM-MM MD : \cpmd\ \cite{CPMD} and \cptk\ \cite{CP2K}
\end{itemize}
\noindent To prepare the input files for these calculations is likely to be the key, and most complicated step towards MD simulations. 
\atomes\ offers a user-friendly assistant to help and guide the user step by step to achieve this crucial step. 
\begin{figure}[!p]
\hypertarget{overview}{
\begin{center}
\includegraphics*[width=23cm, angle=90, keepaspectratio=true, draft=\ddst]{img/overview3}
\end{center}
\caption{Overview of the \atomes\ program.}\label{overview}
}
\end{figure}
