\chapter{Visual edition in \atomes}
\markboth{Visual edition in \atomes}{\scshape \thesection}
\label{edition}

In this chapter the examples will be illustrated using the same workspace used to illustrate the visual analysis 
capabilities of \atomes\ in the previous chapter and presented in figure~\ref{workm}. \\
Edition tools are available via the \aob{Tools} menu, and using either the \aob{Edit} submenu [Fig.~\ref{emf}-a]
or the mouse \aob{Edition} from in the \aob{Mouse mode} submenu [Fig.~\ref{emf}-b]: \\
\emfig

\section{The \aob{Edit} submenu}

\subsection{The \aob{Crystal builder} window}

The \aob{Crystal builder} button [Fig.~\ref{crdw}] allows to open the corresponding window. \\
\myfigure{h}{crdw}{\image{11}{img/edit/builder-m}}{Accessing the \aob{Crystal builder} window in the \atomes\ program.}{Accessing the \aob{Crystal builder} window in the \atomes\ program.}
\newpage
\myfigure{h}{crwin}{
\hspace{-0.5cm}
\image{17.5}{img/edit/builder/builder}}
{The \aob{Crystal builder} window in the \atomes\ program.}
{The \aob{Crystal builder} window in the \atomes\ program.}
\noindent The \aob{Crystal builder} window [Fig.~\ref{crwin}] allows the creation of crystalline structure(s) and super-structure(s). 
Thus not only atom(s) but also molecules can be inserted to the crystalline positions defined by the space group. \\
The \aob{Select ...} menu allows to insert objects: 
atoms, molecules, and atom selections from any other project opened in the \atomes\ workspace. 
It will be presented in details in section~\ref{replsel}, at the exception of the \aob{Empty site} button unique to the \aob{Crystal builder} and that will be introduced in the next pages related to the occupancy. \\
Each object will be inserted at a position specified using the fractional coordinates, in the proportion of the occupancy, and following the rules of the selected space group. 
\clearpage
The following parameters can be adjusted: 
\begin{enumerate}
\item\label{cs} The crystal system:\\[0.25cm]
\begin{minipage}{4cm}
 \begin{itemize}
\item Triclinic
\item Monoclinic
\end{itemize}
\end{minipage}
\begin{minipage}{4cm}
\begin{itemize}
\item Orthorhombic
\item Tetragonal
\end{itemize}
\end{minipage}
\begin{minipage}{4cm}
\begin{itemize}
\item Trigonal
\item Hexagonal
\end{itemize}
\end{minipage}
\begin{minipage}{4cm}
\begin{itemize}
\item Cubic
\end{itemize}
\end{minipage}
\item\label{bv} The type of Bravais lattice, depending on \ref{cs}, among:\\[0.25cm]
\begin{minipage}{4cm}
\begin{itemize}
\item Primitive
\item Base-centered
\end{itemize}
\end{minipage}
\begin{minipage}{4cm}
\begin{itemize}
\item Body-centered
\item Face-centered
\end{itemize}
\end{minipage}
\begin{minipage}{5cm}
\begin{itemize}
\item Hexagonal axes
\item Rhombohedral axes
\end{itemize}
\end{minipage}
\item\label{sp} The space group, from 230 groups of the International tables for Crystallography Vol. A, \cite{IucrA}
and filtered using \ref{cs} and \ref{bv}
\item\label{sps} The space group setting, if more than one is available.
\item\label{lp} The format of the lattice parameters:
\begin{itemize}
\item {\bf{\textit{a}}}, {\bf{\textit{b}}}, {\bf{\textit{c}}} and $\alpha$, $\beta$, $\gamma$
\item Vector components: {\it{a}}(x,y,z), {\it{b}}(x,y,z), {\it{c}}(x,y,z)
\end{itemize}
\item The number of cell(s) to replicate on {\bf{\textit{a}}}, {\bf{\textit{b}}} and {\bf{\textit{c}}} to create the final supercell. 
\item The object to insert to build the crystal using information specified at steps \ref{cs}, \ref{bv}, \ref{sp}, \ref{sps} and \ref{lp},
including the following options:
\begin{itemize}
\item The type of object (atom, molecule ...)
\item The fractional coordinates of the object
\item The occupancy of the crystallographic site (between 0.0 and 1.0)
\item How to handle occupancy (if < 1.0), object(s) and/or empty position(s):
\begin{itemize}
\item \aob{Random for the initial cell only}: sites are filled randomly in the initial cell only, then the initial cell is simply replicated.
\item \aob{Random cell by cell}: sites are filled randomly for each cell, cell by cell separately.
\item \aob{Completely random}: sites are filled randomly for the entire network, the final crystal is considered as whole.
\item \aob{Successively}: sites are filled successively, all object(s) A are inserted (for the first $n(A)$ positions), then all object(s) B are inserted (for the next $n(B)$ positions) ...  
\item \aob{Alternatively}: sites are filled alternatively: object A is inserted on the first position, object B is inserted on the second position, object A on the third position, 
object B on the fourth position ... and so on.
\end{itemize}
In any case the number $n(A)$ of object(s) A to be inserted in calculated using:
\begin{equation} n(A) = NP \times occ(A) \nonumber \end{equation}
$NP$ is the number of position(s) to fill, $occ(A)$ is the occupancy for object A.\\
\underline{Overlapping}, object(s) on the same position, can also be allowed, the concept will be illustrated in the following examples.
\end{itemize}
\end{enumerate}

\clearpage

\subsubsection*{Using the crystal builder: space group information}

Once a space group has been selected it is possible to access the \aob{Space group info} dialog using the \aob{group info} button:
\myfigure{h}{crinfo}{\image{17}{img/edit/builder/info}}
{The \aob{space group info} dialog for the {\bf{R$\bar{3}$c}} group in the \atomes\ program.}{The \aob{space group info} dialog for the {\bf{R$\bar{3}$c}} group in the \atomes\ program.}
\laf The main information available is presented in the \aob{space group info} dialog [Fig.~\ref{crinfo}], including the different setting(s), 
and the corresponding initial coordinates and Wyckoff positions. 

\newpage

\subsubsection*{Using the crystal builder: building crystal(s)}

The following pages will present crystal building examples using the {\bf{Fd$\bar{3}$m}} (origin 1 setting) space group (diamond-like structure):\\
\begin{enumerate}
\item\label{cd1} {\bf{\textit{The C-diamond structure}}}:
\myfigure{h}{diaminit}{\image{17}{img/edit/builder/diam-init}}
{Building a C-diamond crystal in the \atomes\ program.}{Building a C-diamond crystal in the \atomes\ program.}
\laf The following steps are required to build the C-diamond crystalline structure (see figure~\ref{diaminit}-left):
\begin{itemize}
\item Select the {\bf{Fd$\bar{3}$m}} space group (\aob{Cubic -> Face-centered -> 227: Fd-3m})
\item Set the value for the lattice parameter: $\simeq$ 3.5~\AA
\item Insert a C atom at (0.0, 0.0, 0.0) with occupancy to 1.0, you can edit both coordinates and occupancy by double-clicking on the corresponding line.
\item Select the \aob{Insert check button} to use the atom to build the crystal, 
	or alternatively click to top of the "Insert" column to insert all elements.
\item Finally simply click on the \aob{Build / Build (new project)} button.
\end{itemize}
The crystal is then displayed (see figure~\ref{diaminit}-right), also clones (atoms linked using the PBC see section~\ref{Clones}) are immediately shown. 
The existence of a chemical bond depending on the cutoff(s), determined automatically when creating the crystal, 
you might need to adjust the values to define properly the bonding of the system. \\
\newpage
\item {\bf{\textit{The C-diamond structure, playing with occupancy}}}:\\
In these examples the parameters to build the crystal remain exactly similar to \ref{cd1}, only the total number of cells will be increased to 2 on {\bf{\textit{a}}}:
\begin{center}\image{8}{img/edit/builder/diam2cells}\end{center}
\begin{itemize}
\item Occupancy = 1.0: \\
\begin{center}\image{6}{img/edit/builder/diam2a}\end{center}
With 8 atoms per cell, the total number of atoms in the system is equal to 16.  
\item Occupancy = 0.5: \\[0.25cm]
\begin{minipage}{17cm}
\hspace{-2cm}\image{17}{img/edit/builder/occup} \\[0.25cm]
With an occupancy of 0.5 the number of atoms per cell will be reduced to 4,\\
and the total number of atoms in the system to 8.
\end{minipage}\\
\begin{minipage}{16cm}
\hspace{-1cm}
\begin{tabular}{p{0.5cm}p{6cm}p{0.5cm}p{12cm}}
\hspace{-1cm}\raisebox{2.15cm}{(1)} & \hspace{-1cm} \image{6}{img/edit/builder/occ-1} & \hspace{-0.5cm} \raisebox{2.15cm}{$\Longrightarrow$} &
\image{6}{img/edit/builder/diam2a-1} 
\end{tabular}
\aob{Random for the initial cell only}: \\
The same treatment is applied to each cell, inducing an overall symmetry. \\
The system is not disordered with 4 atoms per cell at the exact same positions. \\
\end{minipage}\\
\begin{minipage}{16cm}
\hspace{-1cm}
\begin{tabular}{p{0.5cm}p{6cm}p{0.5cm}p{12cm}}
\hspace{-1cm}\raisebox{2.15cm}{(2)} & \hspace{-1cm} \image{6}{img/edit/builder/occ-2} & \hspace{-0.5cm} \raisebox{2.15cm}{$\Longrightarrow$} &
\image{6}{img/edit/builder/diam2a-2} 
\end{tabular}
\aob{Random cell by cell}: \\
A random treatment is applied to each cell independently. \\
The system is disordered, but the number of atoms per cell remains equal to 4. \\
\end{minipage}\\
\begin{minipage}{16cm}
\hspace{-1cm}
\begin{tabular}{p{0.5cm}p{6cm}p{0.5cm}p{12cm}}
\hspace{-1cm}\raisebox{2.15cm}{(3)} & \hspace{-1cm} \image{6}{img/edit/builder/occ-3} & \hspace{-0.5cm} \raisebox{2.15cm}{$\Longrightarrow$} &
\image{6}{img/edit/builder/diam2a-3} 
\end{tabular}
\aob{Completely random}: \\
The system is entirely disordered. \\
The average value 0.5 C atoms per site is respected for the entire structure, \\
but the number of atom(s) per cell can change. \\
This option if particularly useful when the occupancy for a particular site is very low, see the section remarks afterwards for more information.
\end{minipage}\\
\newpage
To illustrate (4) \aob{Successively} and (5) \aob{Alternatively} it is interesting to use more than one chemical species: \\[0.25cm]
\begin{tabular}{p{8cm}p{1cm}p{8cm}}
\hspace{-3cm}Adding O atoms: & & \hspace{-3cm}Adding O and N atoms: \\
\hspace{-3cm}For all chemical species: & & \hspace{-3cm}For all chemical species: \\
\hspace{-2cm}- The occupancy is equal to 0.5 & & \hspace{-2cm}- The occupancy is equal 0.333 \\
\hspace{-2cm}- The site is the same (0.0, 0.0, 0.0) & & \hspace{-2cm} - The site is the same (0.0, 0.0, 0.0) \\[0.25cm]
	$\Downarrow$ & & $\Downarrow$ \\
\hspace{-3cm}\image{8}{img/edit/builder/diam2cells2}
& &
\hspace{-3cm}\image{8}{img/edit/builder/diam2cells3} \\
	$\Downarrow$ & & $\Downarrow$ \\
\hspace{3cm}\image{6}{img/edit/builder/occ-4} \\
\hspace{-2cm}\image{6}{img/edit/builder/diam2a-4} & &
\hspace{-2.5cm}\image{6}{img/edit/builder/diam2a-43} \\
\\
\hspace{3cm}\image{6}{img/edit/builder/occ-5} \\
\hspace{-2cm}\image{6}{img/edit/builder/diam2a-5} & &
\hspace{-2.5cm}\image{6}{img/edit/builder/diam2a-53} \\
\end{tabular}
\\[0.25cm]In both cases (4) \aob{Successively} and (5) \aob{Alternatively}: \\[0.25cm]
The system is ordered. \\
The average values of the occupancy per site are respected for the entire structure.

\newpage
\item {\bf{\textit{The C-diamond structure, playing with molecules}}}: \\
As already mentioned \atomes\ does not only allow to build atomic crystalline structures but also molecular crystalline super-structures. 
\myfigure{h}{c60build}{\image{16}{img/edit/builder/build-c60}}
{Building a C-diamond like C$_{60}$ crystalline super-structure.}
{Building a C-diamond like C$_{60}$ crystalline super-structure.}
\clearpage
\noindent 
The \aob{Insert...} menu allows indeed to insert fragments from the library or from any project opened in the workspace. \\
Using C$_{60}$ fullerene molecules instead of carbon atoms (and increasing the lattice parameter appropriately) allows to obtain the structure illustrated in figure~\ref{c60build}. \\
As for the atom(s) tweaking the occupancy can be useful with molecules:
\begin{center}\image{8}{img/edit/builder/diam2cells2mols}\end{center}
Using the (4) \aob{Successively} occupancy set-up, you build:
\myfigure{h}{c60-tol}{\image{11}{img/edit/builder/c60-tol}}
{Building a C-diamond like, alternating C$_{60}$ and toluene molecules, crystalline super-structure.}
{Building a C-diamond like, alternating C$_{60}$ and toluene molecules, crystalline super-structure.}
\item {\bf{\textit{The C-diamond structure, molecules and overlapping}}}:\\
As mentioned using the \aob{Allow overlapping} option allows to insert at the same crystalline positions. 
The interest is limited to molecules-molecules overlapping or molecules-atoms overlapping, there is no
point in getting two atoms at the same position. 
The idea behind this concept is to allows the encapsulation of an object by another one:
\myfigure{h}{c240b}{\image{16}{img/edit/builder/build-c240}}
{Building a C-diamond like, toluene encapsulated  C$_{240}$, crystalline super-structure.}
{Building a C-diamond like, toluene encapsulated  C$_{240}$, crystalline super-structure. 
The \aob{Molecules} color-map (see sec.~\ref{csm}) for atoms and bonds is used for clarity purposes.}
\end{itemize}
\end{enumerate}
\clearpage
\noindent{\bf{Notes:}}
\begin{itemize}
\item \uline{Crystal building}: for more on the crystal building process see appendix~\ref{cbp}. \\[0.25cm]
Here are some important things happening when building a crystal in \atomes:
\begin{itemize}
\item At the beginning of the process the size of the object(s) to insert is compared to the lattice parameters, 
if the lattice parameters are considered too small a warning message will pop-up: 
\begin{itemize}
\item For an atom, the size is set to 1.0~\AA.
\item For a molecule, the size is the maximum interatomic distance within the molecule.
\end{itemize}
The process is not fail-safe, and without being careful it is possible to build a crystal with an improper bonding. \\
The distance matrix calculation to determine bonding information is performed on the fly at the end of the building process,
if the crystal bonding is too bad, this calculation could fail and no bond will be display. 
If the crystalline structure is good, then simply correct the bond cutoff to perform the calculation again, 
and get the correct bonding information.  
\item At the end of the process all the atoms are wrapped back in the unit cell. 
\item The bond cutoff are determined automatically, therefore the visual aspect of the final structure might be misleading, 
with too much of too few bonds compared to what was expected.
\item When inserting a molecule remember that there is not way to determine its orientation, with water molecules for instance 
it might be required to rotate the molecule(s) afterwards, see section~\ref{medial} for more information.
\end{itemize}
\item \uline{Occupancy}:
as illustrated previously the order the object(s) is(are) inserted on the crystalline positions might have an importance on the final structure. 
This is true when the occupancy is < 1.0 and / or when object can share the same site when using the \aob{Allow overlapping} option. \\
To ensure that the desired crystal will be built, either insert the object in the proper order or re-order them using the mouse drag an drop:
\begin{center}\image{10}{img/edit/builder/dnd}\end{center}
The occupancy $occ({\bf{\textit{s}}})$ for a site {\bf{\textit{s}}} $(x_s, y_s, z_s)$ is defined as: 
\begin{equation}occ({\bf{\textit{s}}}) = \sum_{i=1}^{N} occ(i)
\end{equation}
for the $N$ object(s) on the same site $(x_s, y_s, z_s)$
\item \uline{Empty position}: 
the \aob{Empty position} button of the \aob{Select ...} menu is used to force the insertion of empty position(s). \\
The empty site(s) are always treated before any object insertion(s), even if the position in the insertion tree is not on top. \\
In the case of \aob{Overlapping}+\aob{Empty position} the rules for occupancy are modified as follow: \\
\begin{equation}
occ({\bf{\textit{s}}}) = \max_{i=1}^{N} occ(i) + \sum_{j=1}^{E} occ(j) \\[0.25cm]
\end{equation}
for the $N$ object(s) and $E$ empty positions on the same site $(x_s, y_s, z_s)$.
\end{itemize}

\subsection{The \aob{Cell edition} window}

The \aob{Cell edition} window allows to adjust many parameters related to the periodicity of the material, 
and is accessible using any of the buttons of the \aob{Cell} submenu [Fig.~\ref{cedw}]. \\
\myfigure{h}{cedw}{\image{11}{img/edit/cell-m}}
{Accessing the \aob{Cell edition} window in the \atomes\ program.}{Accessing the \aob{Cell edition} window in the \atomes\ program.}
\laf Each of the buttons in the \aob{Cell} menu [Fig.~\ref{cedw}] allows to open the corresponding tab in the \aob{Cell edition} window. \\
Each of these tabs and the action they provide will be introduced in the following: 
\begin{itemize}
\item The actions of the \aob{Cell edition} window are accessible if and only if there is a model box. 
\item The actions \aob{Create super-cell}, \aob{Adjust density} and \aob{Cut slab} are not available in the case of MD trajectory.
\end{itemize}
\newpage
\subsubsection*{Wrap all atoms in}
The \aob{Wrap all atoms in} button allows to wrap all atomic coordinates in the original unit cell. 
\begin{enumerate}
\item As illustrated bellow MD codes sometimes output MD trajectory / atomic coordinates in real coordinates, 
the visual assessment of such trajectory / atomic coordinates becomes complicated:
\begin{center}\image{12}{img/edit/cell/l-GeS2-nw}\end{center}
\clearpage
\item To simplify the visual analysis \atomes\ can put back all atoms back in the unit cell using the periodicity and the box
parameters as described in the \aob{Box and periodicity} dialog [Fig.~\ref{bpd}]. \\
Simply select the \aob{Wrap atomic coordinates in unit cell} menu item from [Fig.~\ref{cedw}]:  
\item[] The action being irreversible it is required to confirm it:
\begin{center}\image{9}{img/edit/cell/wrap_yn}\end{center}
\item After confirmation the operation is performed and the OpenGL window updated:
\begin{center}\image{12}{img/edit/cell/l-GeS2-yw}\end{center}
\end{enumerate}
\newpage
\subsubsection*{Shift center}

The \aob{Shift center} tab allows to shift the atomic coordinates within the unit cell: \\[0.5cm] 
\begin{tabular}{cp{0.5cm}c}
\hspace{-1cm}
 & & \multirow{3}{6cm}{\image{9}{img/edit/cell/shift/Ni-Phth}} \\
 \image{6.5}{img/edit/cell/shift/shift-0} & \raisebox{2.0cm}{$\Longrightarrow$} \\
 \\[1.5cm]
 & $\Swarrow$ \\
 & & \multirow{3}{6cm}{\image{9}{img/edit/cell/shift/Ni-Phth-s}} \\
 \image{6.5}{img/edit/cell/shift/shift-y} & \raisebox{2.0cm}{$\Longrightarrow$} \\
\end{tabular}
\\[2.5cm]
It is possible to shift atomic coordinates along the {\em{x}}, {\em{y}} and/or {\em{z}} model axis. 
The periodicity is preserved and if needed bonding properties are re-evaluated on the fly during this operation. \\
\\
\noindent{\bf{Note that it this required to wrap the atomic coordinates in the unit cell before being able to shift them.}}
 
\clearpage

\subsubsection*{Add extra(s)}

\begin{tabular}{cp{1.0cm}c}
 & & \multirow{3}{6cm}{\image{7}{img/edit/cell/extra-sup/g-SiO2-311}} \\
 \image{6}{img/edit/cell/extra-sup/cadex-311} & \raisebox{1.0cm}{$\Longrightarrow$} \\
 \\[3cm]
 & $\Swarrow$ \\
 & & \multirow{3}{6cm}{\image{7}{img/edit/cell/extra-sup/g-SiO2-331}} \\
 \image{6}{img/edit/cell/extra-sup/cadex-331} & \raisebox{1.0cm}{$\Longrightarrow$} \\
 \\[3cm]
 & $\Swarrow$ \\
 & & \multirow{3}{6cm}{\image{7}{img/edit/cell/extra-sup/g-SiO2-333}} \\
 \image{6}{img/edit/cell/extra-sup/cadex-333} & \raisebox{1.0cm}{$\Longrightarrow$}
\end{tabular}

\clearpage

\noindent The \aob{Add extra(s)} tab allows to add extra unit-cell(s) to the model, any object visible within the original unit cell will be duplicated 
(atom, bond, polyhedra, measure, label ...) as well. The duplicates cells are slightly translucent compared to the initial unit cell.

\subsubsection*{Create super-cell}

The \aob{Create super-cell} button allows to change the periodicity of the system 
and transform the unit cell to the extended structure created after using the \aob{Add extra(s)} tab. \\
\begin{center}\image{10}{img/edit/cell/extra-sup/g-SiO2-333-sc}\end{center}

\subsubsection*{Adjust density}
The \aob{Adjust density} tab allows to modify the density of the material using homothetic rescaling that can be:
\begin{itemize}
\item Homogeneous: the $\frac{a}{b}$ and $\frac{a}{c}$ ratio of the initial unit cell are preserved 
\item Heterogeneous: a, b, and c are adjust individually
\end{itemize}
Note that to perform this operation any bonding information will be lost, and only the visual information will not be immediately erased. 
Therefore it is strongly recommended to recompute the bonding proprieties afterwards, this can not be done on the fly because
changing the density is likely to require to change the bond cutoffs as well.
\clearpage
\begin{tabular}{cp{1.0cm}c}
 & & \multirow{3}{6cm}{\image{7}{img/edit/cell/dens/g-SiO2-m}} \\
 \image{6}{img/edit/cell/dens/cell-25} & \raisebox{2.0cm}{$\Longrightarrow$} \\
 \\[2cm]
 $\Uparrow$ & \\
 & & \multirow{3}{6cm}{\image{7}{img/edit/cell/dens/g-SiO2-0}} \\
 \image{6}{img/edit/cell/dens/cell-35} & \raisebox{2.0cm}{$\Longrightarrow$} \\[0.5cm]
 $\Downarrow$ \\[2.0cm]
 & & \multirow{3}{6cm}{\image{7}{img/edit/cell/dens/g-SiO2-p}} \\
 \image{6}{img/edit/cell/dens/cell-45} & \raisebox{2.0cm}{$\Longrightarrow$}
\end{tabular}

\clearpage

\subsubsection*{Cut slab}

The \aob{Cut slab} tab allows in the model using geometric patterns: parallelpiped, cylindrical and spherical. 
The \aob{Cut slab} tab displays slab information, including slab volume and the number of atom of each chemical species inside it. 
The values are refreshed each time the shape, position size, and/or rotation of the slab is modified. 
Examples are presented in figure.~\ref{excut}. \\
\myfigure{h}{excut}{\image{18}{img/edit/cell/cut/global}}
{Model edition using the \aob{Cut slab} tab.}
{Model edition using the \aob{Cut slab} tab, from left to right: 1) all atoms outside the sphere are deleted, 2) atoms inside the cylinder are selected, 3) all atoms in the parallelpiped slab are deleted and the results is exported in a new project.}
%\begin{tabular}{lcp{0.25cm}lcp{0.25cm}lc}
%\hspace{-3cm} {\bf{a)}} & & & {\bf{b)}} & & & {\bf{c)}} \\
%\hspace{-3cm} & \image{6}{img/edit/cell/cut/slab-par} & & 
% & \image{6}{img/edit/cell/cut/slab-cyl} & & 
% &\image{6}{img/edit/cell/cut/slab-sph} \\[0.25cm]
%\hspace{-3cm} & $\Downarrow$ & & & $\Downarrow$ & &  & $\Downarrow$ \\[0.25cm]
%\hspace{-3cm} & \image{5}{img/edit/cell/cut/sio2-par-tab2} & & 
% & \image{5}{img/edit/cell/cut/sio2-cyl-tab2} & & 
% &\image{5}{img/edit/cell/cut/sio2-sph-tab2} \\
%\end{tabular}
%\item Once the slab is ready, it is possible either to select / cut / export as new project the atomes inside or outside the slab: \\
%\begin{tabular}{lcp{0.25cm}lcp{0.25cm}lc}
%\hspace{-3cm} {\bf{a)}} & & & {\bf{b)}} & & & {\bf{c)}} \\
%\hspace{-3cm} & \image{5}{img/edit/cell/cut/sio2-par-cut} & &
% & \image{5}{img/edit/cell/cut/sio2-cyl-cut} & & 
% &\image{5}{img/edit/cell/cut/sio2-sph-cut} \\
%\end{tabular}

\subsection{The \aob{Model edition} window}
\label{medial}

The \aob{Model edition} window offers several tools dedicated to the edition of the atomic coordinates 
and is accessible using any of the buttons of the \aob{Atom} submenu [Fig.~\ref{atedw}]. \\
\myfigure{h}{atedw}{\image{11}{img/edit/atom-m}}{Accessing the \aob{Model edition} window in the \atomes\ program.}{Accessing the \aob{Model edition} window in the \atomes\ program.}
\laf It is not possible to access the \aob{Model edition}, and therefore to move / replace / remove and insert atom(s) 
in the case of MD trajectory. 

\subsubsection*{Selection process}
\label{newsel}

When the \aob{Atom edition} window is opened the main OpenGL window remains active, and it is still possible to work in it. 
However having the \aob{Model edition} window opened will modify the results of the atom selection process, 
whether is it performed using any dialog window or using with the mouse describe in section~\ref{msel} and \ref{mobjm}. 
The newly extra-selected atom(s) will be covered in pink (instead of light blue), see [Fig.~\ref{v2sel}], 
and even already selected atoms (in light blue) can be re-selected. 
\eselfig
\laf As long as the \aob{Model edition} window remains open, the \aob{Measures} dialog [Fig.~\ref{mdial}] will only presented measurement(s) related to the new pink atoms. \\
The utilization and purpose of this extra selection feature will be illustrated when presenting the action tabs of the \aob{Model edition} window. 

\subsubsection*{General behavior}

Before browsing each and every tab of the \aob{Model edition} window it is required to introduce the \aob{atom search} tool [Fig.~\ref{atst}] 
common to almost every tab (in the top part) in the window. 
Each tab being dedicated to a particular action the search tool will help to find and selected the atom(s) you want the action to be performed upon: \\
\myfigure{h}{atst}{\image{15}{img/edit/atoms/search}}{The atom(s) search tool.}{The atom(s) search tool.}
\begin{enumerate}
\item\label{sel} Search for atom(s) among [Fig.~\ref{atst}-1]: 
\begin{enumerate}
\item non-selected atom(s) only: search only in \aob{All non-selected atoms} [Fig.~\ref{atst}-1] 
\item\label{selao} selected atom(s) only: search only in \aob{All selected atoms} [Fig.~\ref{atst}-1] 
\item\label{allao} or all atoms: search in all \aob{All atom(s)} [Fig.~\ref{atst}-1] 
\end{enumerate}
\ref{selao} is the default value for menu [Fig.~\ref{atst}-1] if some atoms are selected, and \ref{allao} is the default value otherwise. 
\item\label{for} Apply the action to [Fig.~\ref{atst}-2]:
\begin{enumerate}
\item\label{isoat} The atoms as single object(s): pick \aob{Atom(s)} [Fig.~\ref{atst}-2]
\item\label{groat} Group of atom(s): pick \aob{Group of atoms} [Fig.~\ref{atst}-2]
\end{enumerate}
In case \ref{isoat} the action will be performed on the atom(s) that will be selected and each of these atoms will be treated as an isolated object. 
In case \ref{groat} the action will be performed on the group(s) of atoms, the atom(s) that will be selected afterwards belong to: entire coordination sphere(s), 
fragment(s) or molecule(s).
\item\label{fil} Filter the result of the search depending on the type of object you are interested in figure~\ref{atst}-3:
\begin{enumerate}
\item Chemical species (Available only if \ref{isoat} is selected for [Fig.~\ref{atst}-2])
\item Total coordination 
\item Partial coordination
\item Fragment
\item Molecule
\end{enumerate}
Pick the appropriate value in figure~\ref{atst}-3 and the selection tree bellow will be updated accordingly.
\item Refine the search to present only the atom(s) of that particular chemical species (if \ref{isoat} was selected in figure~\ref{atst}-2) 
or the group of atoms that contains such chemical species (if \ref{groat} was selected in figure~\ref{atst}-2)
\end{enumerate}
{\bf{Notes}}:
\begin{itemize}
\item \uline{Clickable} : some of the column headers in the search tree are clickable, in the top part if you click on the \aob{Label} or \aob{Move} the corresponding action will be applied / unapplied to the entire data set available in the search tree. 
\item \uline{Update} : whenever the values for the menus~\ref{for} [Fig.~\ref{atst}-2] and \ref{fil} [Fig.~\ref{atst}-3] are modified 
the tree is not only refreshed but also cleaned, and so is the corresponding data thus previous selection sets would be lost in the process.
\end{itemize}
Examples of the utilization of the \aob{atom search} tool [Fig.~\ref{atst}] will be presented along with each actions available in the \aob{Model edition} window. 

\subsubsection*{The \aob{Move} tab}
\label{movetab}
The \aob{Move} tab allows to move atom(s) (translation only) or group of atoms (translation and rotation around the groups barycenter).  
In the following example one of the fragment in the model is selected, the fragment appears to be split because of the periodic boundary conditions: \\
\myfigure{h}{nipinit}{\image{15}{img/edit/atoms/move/Ni-Phth-init}}{Initial position with an isolated fragment selected in the model.}{Initial position with an isolated fragment selected in the model.}
\laf To move the selected fragment:
\begin{enumerate}
\item Open the \aob{Move} atom(s) tab: 
\begin{center} \image{13}{img/edit/atoms/move/move-tab} \end{center}
\newpage
\item Go back to the main OpenGL window, keeping the \aob{Model edition} window opened, and select few atoms, that will appear in pink color. 
Then open the \aob{Measures} dialog [Fig.~\ref{mdial}] to select a bunch of measurements to be displayed: 
\begin{center}\image{16}{img/edit/atoms/move/show-meas}\end{center}
\item Go back to the \aob{Model edition} window and adjust the search \aob{For} option and select \aob{Group of atom(s)}: 
\begin{center}\includegraphics*[height=4cm, keepaspectratio=true, draft=\ddst]{img/edit/atoms/move/sel-group}\end{center}
\item Adjust the search \aob{Filter by} option and select \aob{Fragment}: 
\begin{center}\includegraphics*[height=4cm, keepaspectratio=true, draft=\ddst]{img/edit/atoms/move/sel-frag}\end{center}
\newpage
\item Then select the fragment in the selection tree: 
\begin{center}\image{13}{img/edit/atoms/move/move-2}\end{center}
\item Open the lower part(s) to move the fragment:
\begin{center}\image{13}{img/edit/atoms/move/motion}\end{center}
\end{enumerate}
\myfigure{h}{nipmoved}{\image{15}{img/edit/atoms/move/Ni-Phth-movedm}}
{Final position of the fragment, after translation and rotation in the model.}
{Final position of the fragment, after translation and rotation in the model.}
\noindent {\bf{Notes:}}
\begin{itemize}
\item \uline{Motion axis}: it is possible to select (and visualize) the motion axis and choose between the \aob{Eye (viewer) axis} and the \aob{Model axis} (real x, y and z of the atoms).
\item \uline{Motion actions}: the motion interactors (ranges and input entries) allow to adjust motion very precisely, slightly more approximate motion is also available using the mouse (see Sec.~\ref{mouseedit})
\item \uline{Measurements}: if already present in the OpenGL window, measurement(s) are updated on the fly, allowing to adjust perfectly position and orientation 
of the object in the model, see [Fig.~\ref{nipmoved}].
\item \uline{Motions}: translation is always available, however rotation is only available for \aob{Group of atom(s)} and is performed around the barycenter(s)
of the object(s) to be rotated.
\item \uline{Reconstruction}: the check button labelled \aob{Extract/rebuild the object(s) to be moved, ie. cut/clean bonds with nearest neighbor(s)} offers 
the following option:
\begin{itemize}
\item For \aob{Atom(s)}: if activated the object(s) to be moved will be extracted from the model and translated independently, otherwise chemical bond(s) will simply be stretched. 
\item For \aob{Group of atom(s)}: if activated the atomic positions of the object(s) will be corrected to get only single piece object(s). 
\end{itemize}
\newpage
As illustrated in figure~\ref{nipinit} the fragment visually appears in 2 pieces, and as illustrated by the example as a single piece object after motion [Fig.~\ref{nipmoved}]. \\
If the object(s) to be moved are coordination spheres and if the spheres that are being moved share atom(s), then the positions of these atoms 
might be affected twice by this procedure, leading to an awkward, yet correct, 3D representation.
\item \uline{Bonding}: During motion, the coordination information is modified, all related menus and corresponding dialogs are updated accordingly, 
however detailed bonding information is lost. It is therefore strongly recommended to recompute bonding properties afterwards. 
\end{itemize}

\clearpage

\subsubsection*{The \aob{Replace} tab}
\label{replsel}

The \aob{Replace} tab allows to replace object(s), atom(s) or group of atoms, by either: atom(s), 
molecules imported from the internal library or atom(s) imported from any project opened within the same instance of the \atomes\ program:  
\myfigure{h}{repinit}{\image{13}{img/edit/atoms/replace/repl}}
{The \aob{Replace} tab with the option of replace objects \aob{Normally} or \aob{Randomly}.}
{The \aob{Replace} tab, with the option of replace objects \aob{Normally}, ie. object by object, or \aob{Randomly}, in the model.}
\laf As illustrated in figure~\ref{repinit} it is possible to replace object(s) one by one (\aob{Normally}) or by random pick (\aob{Randomly}). 
The difference between the normal and the random search trees are illustrated in top part of figure~\ref{repnm}. \\
For normal substitution(s) [Fig.~\ref{repnm}-a] it is possible to select the replacement object(s) either at once for all objects, using the \aob{Select for all ...} menu, 
or alternatively one by one when browsing the tree and using the \aob{Select ...} menu(s). \\
For random substitution(s) [Fig.~\ref{repnm}-b] only a single replacement is selected for each type of object, and it is require to enter the number of substitution(s) to be performed. \\
The substitution options accessible using the \aob{Select ...} menu are illustrated in the bottom part in figure~\ref{repnm} with the different \aob{atoms} c), \aob{Library} d) and \aob{Import from project} e) submenus. 
\clearpage
\repnmfig

\noindent {\bf{Substitution options}}:
\begin{itemize}
\item The \aob{Atom} submenu [Fig.~\ref{repnm}-c], to insert a new atom in place of the object(s) to be removed. 
Some shortcuts are proposed, if required the \aob{Other ...} button allows to open a periodic table to pick any appropriate atom: 
\begin{center}\image{14}{img/edit/atoms/perio}\end{center}
\item The \aob{Library} submenu [Fig.~\ref{repnm}-d], to insert an new molecule in place of the object(s) to be removed. 
Some shortcuts are proposed as well, and the \aob{More ...} button allows to open the \aob{Library} dialog:
\begin{center}\image{12}{img/edit/atoms/library}\end{center}
The \aob{Library} dialog provides a bunch of sample molecules roughly sorted by chemical properties (see appendix \ref{lib} for more details).
\item The \aob{Import from project} submenu [Fig.~\ref{repnm}-e], to insert atom(s) from any \atomes\ project opened in the workspace. 
\item The \aob{Copied data} button [Fig.~\ref{repnm}-f], to copied data selection from any workspace in \atomes. 
\end{itemize}
\noindent An example is provided in figure~\ref{repex} with the random substitution of 500 \aob{Si} atoms from the \aob{\sio} project by \aob{water} molecules. \\ 
\myfigure{h}{repex}{\image{16.5}{img/edit/atoms/replace/g-SiO2-repex}}
{Random substitution of 500 \aob{Si} atoms by \aob{water} molecules.}
{Random substitution of 500 \aob{Si} atoms by \aob{water} molecules.}
\clearpage
\noindent{\bf{Notes:}}
\begin{itemize}
\item \uline{Irreversible}: the replacement is irreversible (at least not directly, but it is always possible to perform another backward substitution)
therefore remember to save your work before replacing any object(s) in your model. 
\item \uline{Selection}: At the end of the substitution process, and therefore after the insertion of the new objects, 
all these new objects will appear surrounded by light blue color in the OpenGL window (see figure~\ref{repex}), and will therefore be selected, 
making further work on these new objects easier.  
\item \uline{Barycenters}: the new object is inserted and centered at the barycenter of the atomic positions of the old object to be removed.
\end{itemize}

\subsubsection*{The \aob{Remove} tab}

The \aob{Remove} tab allows to remove object(s), atom(s) or group of atoms, like for the \aob{Replace} tab the action can also be performed randomly. 
An example is provided in figure~\ref{remex} with the normal (one by one) removal of all \aob{Ni-[N$_4$]} coordination spheres atoms from the \aob{Ni-Phth} project: 
for a Nickel atom, this means that this \aob{Ni} atom as well as it surrounding \aob{N} neighbours are all removed. 
\myfigure{h}{remex}{\image{17}{img/edit/atoms/removex}}
{Normal removal of all \aob{Ni-[N$_4$]} coordination spheres from the model.}{Normal removal of all \aob{Ni-[N$_4$]} coordination spheres from the model.}

\subsubsection*{The \aob{Insert} tab}

The \aob{Insert} tab allows to insert new object(s) in the model: 
\myfigure{h}{insinit}{\image{14}{img/edit/atoms/insert/insert}}
{The \aob{Insert} tab, with the options to choose the object to be inserted and the position of insertion.}
{The \aob{Insert} tab, with the options to choose the object to be inserted and the position of insertion.}
\laf The \aob{Select ...} menu reproduces the one from the \aob{Replace} tab, see the bottom part in figure~\ref{repnm},
and allows to insert atom(s) [Fig.~\ref{repnm}-c], molecular fragment(s) [Fig.~\ref{repnm}-d] and atom(s) from any other project opened in the workspace [Fig.~\ref{repnm}-e]. 
Selecting any object to insert will populate and update the tree bellow, with a new line. 
Each line describes the object to be inserted, and offers options to confirm the insertion and specify the position where to insert the object in the model. \\
An example is provided in figure~\ref{insex} where a fullerene \aob{C$_{240}$} is inserted in the \aob{Ni-Phth} project modified after the removal example illustrated in figure~\ref{remex}.
\newpage
\myfigure{h}{insex}{\image{17}{img/edit/atoms/insert/insertex}}
{Insertion of fullerene \aob{C$_{240}$} in the \aob{Ni-Phth} project modified after the removal example.}
{Insertion of fullerene \aob{C$_{240}$} in the \aob{Ni-Phth} project modified after the removal example illustrated in figure~\ref{remex}.}
\noindent{\bf{Notes:}}
\begin{itemize}
\item \uline{Irreversible}: the insertion is irreversible (at least not directly, but it is always possible to remove all inserted atoms)
therefore remember to save your work before inserting any object(s) in your model. 
\item \uline{Selection}: After the insertion of the new objects, 
all these new objects will appear surrounded by light blue color (see figure~\ref{insex})in the OpenGL window, and will therefore be selected, 
making further work on these new objects easier.
\item \uline{Position}: the coordinates (0.0, 0.0, 0.0) always refers to the center of the model, ie. the barycenter of all existing atomic coordinates.
\end{itemize}

\clearpage

\subsubsection*{The \aob{Random move} tab}

The \aob{Random move} tab allows to disorder the model: \\
\myfigure{h}{raninit}{\image{15}{img/edit/atoms/random/ranmove}}
{The \aob{Random move} tab.}
{The \aob{Random move} tab. With \aob{atom(s)} selected as object(s) it is only possible to translate randomly, with \aob{Group of atoms} it is also possible to rotate randomly and the corresponding column appears in the search tree.}
\laf An example is provided in figure~\ref{ranex} with the random rotation of \aob{water} molecules in the modified \aob{\sio} project after the substitution example illustrated in figure~\ref{repex}. Indeed after the substitution \aob{water} molecules were all perfectly aligned, to correct this unlikely organization \atomes\ can rotate randomly and independently every water molecule in the model:
\begin{enumerate}
\item\label{ran1} After the substitution all the \aob{H} and \aob{O} atoms of the newly inserted \aob{water} molecules are selected, so search for selected atom(s) only [Fig.~\ref{atst}-1]. 
\item\label{ran2} Then choose to apply the action to \aob{Group of atoms} [Fig.~\ref{atst}-2].
\item\label{ran3} Filter the selection using the \aob{Fragment} option [Fig.~\ref{atst}-3].
\item After the steps~\ref{ran1}, \ref{ran2} and \ref{ran3} the selection tree will correspond to the one in figure~\ref{ranex}. To enable the random rotation simply click on the top header of the \aob{Rotate} column to select all object(s), actually all \aob{water} molecules. It is required to enter a maximum MSD for each motion, click the top header of the \aob{MAX. MSD} column to enter a suitable value. Optionally it is possible to iter the rotation process, at each step the rotation will be completely random. 
\end{enumerate}
\myfigure{h}{ranex}{\image{16.5}{img/edit/atoms/random/ranmovex}}
{Random rotation of \aob{water} molecules in the modified \aob{\sio} project.}
{Random rotation of \aob{water} molecules in the \aob{\sio} project modified after the substitution example illustrated in figure~\ref{repex}.}
\newpage
\noindent{\bf{Notes:}}
\begin{itemize} 
\item \uline{Group of atoms}: trying to move group of atoms that share atom(s) (ex: 2 coordination spheres with atom(s) in both coordination spheres) can lead to inaccurate results, 
the shared atom(s) being moved with each object they belong to.
\item \uline{Rotation}: the rotation is performed around the positional barycenter of the object's atomic coordinates. 
In the case of a repeated rotation the rotation center is preserved until the final step to avoid drifting. 
If both translation and rotation are repeated then the object is translated first, updating as well the position of the rotation center, and then rotated.
\item \uline{Reconstruction}: the check button labelled \aob{Extract/rebuild the object(s) to be moved, ie. cut/clean bonds with nearest neighbor(s)} offers 
the following option:
\begin{itemize}
\item For \aob{Atom(s)}: if activated the object(s) to be moved will be extracted from the model and translated independently, otherwise chemical bond(s) will simply be stretched. 
\item For \aob{Group of atom(s)}: if activated the atomic positions of the object(s) will be corrected to get only single piece object(s). 
\end{itemize}
\end{itemize}

\clearpage

\subsection{The \aob{Extract/rebuild} buttons}
As introduced in the previous pages these buttons allows to extract object(s) to be moved or copied/pasted from the model, into single piece object(s). 
The effect on motion is illustrated in sections~\ref{movetab} and \ref{asel}, the effect on copy is the following:
\myfigure{h}{rbc}{\image{16.5}{img/edit/copy}}
{Illustration of the rebuild process during copy.}
{Illustration of the rebuild process, a fragment is copied from the top figure 
and pasted back in new projects. That fragment appears split because of the PBC: on the left the fragment is rebuild as a singe object, on the right the 2 pieces remain separated.}

\clearpage

\section{Mouse interaction with the OpenGL window: edition}
\label{mouseedit}

It is possible to activate the mouse \aob{Edition} using the \aob{Tools} menu: 
\editoolfig
\laf It is also possible to use the \Alt+\keystroke{e} keyboard shortcut, and to switch back to \aob{Analysis} mode using the \Alt+\keystroke{a} shortcut. \\
%\begin{figure}[!h]
%\begin{tabular}{ccc}
%\image{7}{img/edit/mouse/am} & \raisebox{0.75cm}{$\Longrightarrow$} &
%\image{7}{img/edit/mouse/em}
%\end{tabular}
%\caption{\label{editool} Activating the mouse \aob{Edition} mode using the \aob{Tools} menu.}
%\end{figure}
\noindent In \aob{Edition} mode the mouse button functions are the following:
\begin{itemize}
\item {\bf{Left button}}
\begin{itemize}
\item {\em{Single click on object}}: object selection
\item {\em{Pressed on background + motion}}: {\bf{selected}} ({\em{only}}) atomic coordinates rotation $^*$
\end{itemize}
\item {\bf{Scroll button}}
\begin{itemize}
\item {\em{Scrolled}}: zoom in/out
\item {\em{Pressed + motion}}: {\bf{selected}} ({\em{only}}) atomic coordinates translation
\end{itemize}
\item {\bf{Right button}}
\begin{itemize}
\item {\em{Pressed on background}}: edition contextual menu
\item {\em{Pressed on object}}: object edition contextual menu
\end{itemize}
\end{itemize}
$*$ The rotation is performed around the coordinates barycenter of the selected atoms.

\subsection{Atom selection}
\label{asel}
As for the \aob{Analysis} mode object(s) can be selected using the mouse left click. 
In \aob{Edition} mode selected atoms/objects are subjects to motion interactions using the mouse and it becomes complicated 
to check motion information with visual measurements. However using the \aob{Tools} menu it is possible to activate the \aob{Measures} selection mode (available when using 
the \aob{Model edition} dialog (see section~\ref{newsel}): \\
\myfigure{h}{editmeas}{\image{10}{img/edit/mouse/selmeas}}
{Activating the \aob{Measures} selection mode using the \aob{Tools} menu.}{Activating the \aob{Measures} selection mode using the \aob{Tools} menu.}
\laf The following provides an example of utilization of the \aob{Measures} selection mode:
\begin{enumerate}
\item Select a fragment in the model: 
\begin{center}\image{11}{img/edit/mouse/movex/Ni-Phth-m0}\end{center}
\newpage
\item Switch to \aob{Edition} mode (if not done already) and change the \aob{Mouse selection mode} to \aob{Measures (Edition mode only)}.
Then select some atoms in the model, and open the \aob{Measures} dialog (see section~\ref{mdw}) to display some measurements:
\begin{center}\image{11}{img/edit/mouse/movex/Ni-Phth-m1}\end{center}
\item Start to move the selected fragment using the mouse left click+motion to rotate or mouse scroll click+motion to translate, 
by default, and if divided because of the PBC, \atomes\ will reconstruct the fragment (see figure~\ref{rccm} to modify that behavior): 
\begin{center}\image{11}{img/edit/mouse/movex/Ni-Phth-m2}\end{center}
\item You will see that when moving the fragment with the mouse the information displayed (inter-atomic distances and angles) 
is instantaneously refreshed:
\begin{center}\image{11}{img/edit/mouse/movex/Ni-Phth-m4}\end{center}
\end{enumerate}

\subsection{The edition contextual menu}

The mouse right click button on the background of the OpenGL window opens the edition contextual menu: \\
\rccmfig
\begin{itemize}
\item The \aob{Tools} submenu [Fig.~\ref{rccm}-a] reproduces the corresponding top bar menu. 
\item The \aob{Insert} submenu [Fig.~\ref{rccm}-b] offers shortcuts to insert objects in the model,
and is similar to the \aob{Select ...} menus described in section~\ref{medial} for the \aob{Replace} and \aob{Remove} tabs of the \aob{Model edition} dialog.
\item\label{erbt} The \aob{Extract/rebuild on motion} button:
\begin{itemize}
\item For \aob{Atom(s)}: if activated the object(s) to be moved will be extracted from the model and translated independently, 
otherwise chemical bond(s) will simply be stretched. 
\item For \aob{Group of atom(s)}: if activated the atomic positions of the object(s) will be corrected to get only single piece object(s). 
\end{itemize}
\item The \aob{Reset motion} button will reset all atomic coordinates to the value immediately saved when:
\begin{itemize}
\item Entering the \aob{Edition} mode: {\bf{if no atoms were inserted or removed in the model.}}
\item Replacing/removing/inserting atom(s): {\bf{if any atom was inserted or removed in the model.}}
\end{itemize}
\end{itemize}

\subsection{The object edition contextual menu}
\label{mobje}
Pressed over any atom/bond the right button opens the object edition contextual menu: 
\oecmfig
\laf The object edition contextual menu in figure~\ref{oecm} follows exactly the construction described in section~\ref{mobjm} 
and \ref{mocmcons} for the object contextual menu in \aob{Analysis} mode. \\
The object edition contextual menu offers shortcuts to \aob{Remove}, \aob{Replace} and \aob{Copy} object(s) in the model.

\clearpage

\section{Keyboard shortcuts}

\kbdedit
